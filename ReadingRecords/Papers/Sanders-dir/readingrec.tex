
\documentclass[11pt]{article}

\usepackage[T1]{fontenc}
\usepackage[utf8]{inputenc}

\begin{document}

\title{Sanders L.(dir.), 2017, Peupler la Terre. De la préhistoire a l’ère des métropoles, Tours, PUFR,  528p.}

\author{J. Raimbault}

\date{}

\maketitle

\begin{abstract}
	This book is the outcome of an interdisciplinary research project with the ambition of understanding transitions of human settlements.
\end{abstract}

% structure of reading record
% (from Clem, West Scale)
% - general introduction and contextualisation
% - description of contents : linear by chapters
% - critics with arguments and references 
% - critical conclusion

% Thibault : presentation generale ; lecture par concepts revisités (transversal) ; conclusion.

% mise en contexte, description du positionnement de l'ouvrage et de son économie générale, type de lectorat ciblé
% la revue s'adresse principalement à des géographes ou à des spécialistes de questions spatiales/territoriales
%  entrées transversales possible ; pas tous chapitres obliges

	
	
%%%%%%
%% Structure

% 35 chercheurs au total	, projet ambitieux sur le plan de l'interdisciplinarite (premier avec autant de disciplines ?)

% 3 parties : def du concept de transition ; dvlpmt des case studies (transitions) ; prise de recul (reflexions transversales et epistemo)


%%%%%%
% Partie I :

%%%
% chap 1 :
% -> etat de l'art sur les transitions. Q chgt etat ou chgt de systeme : non stationarite - au coeur de la difficulte de l'etude des systemes sociaux - ce que certains ont du mal a integrer (cf disc niveaux de stationarite) [rq : quid suites livre blanc ?]
	
% double transfert (// loi )	

% convergence constructivisme et complexite en physique (Thom, chaos, bifurcations)

% specificite des concepts des sciences sociales.

% simpop as synergetic models ? -> dont agree fully.

% transition <=> chgt de regime. / etat alternatif

% info : formalisation du chgt via uml.

% !! endogene / exogene = egalement au coeur du pb de la non stationarite.
% resilience / echelle / systeme collapse


% rq : link with numerous crucial notions.

% cl : particularity of human systems.




%%%%%
% Chap 2 :
% -> def de la transition

% - Def systeme de peuplement
%  archeo : patterns (configuration) vs systems (can be nomadic)
%. geo : set of inhabited kernels

% Def regime : forme, intensite, structure des relations entre entites du systeme = regime (caracterisent le "mode de fonctionnement")
% in physicists term, it would be the topology of the network ; in dynsys the equations ; in ABM the rules (and parameters)
% dynamique des entites n'est pas directement liee au regime.

% besoin d'un cadre conceptuel pour partager et donc comparer le concept de transition.
% lien avec la systemogenese de FDD.

% Rq : relativement reflexif dans la def via l'ontologie p65

% pas base de donnees mais expertise de thematiciens.
% IDEE : utiliser Shesat ?

% Rq : pourquoi pas de def formalisee ? [fig 5 p73 semble pproposer une ?]

% Q des indicateurs

% Presentation des 12 cas d'etude
% 4 mondiaux : sortie d'afrique ; littoralisation ; emergence des villes ; MCR
% 4 regionaux : Bantu ; polarisation europe ; systemes urbains villes-transport (France - co-evolution) ; peuplement sudafriquie.
% RQ : cas de terrain un peu lie a "qui on connait" : dans quelle mesure n'importe quoi aurait pu y rentrer ? et du coup le concept commun serait creux ? [de la non substancialite des concepts ? // sys review resilience // transitions -> TODO]
% 4 "micro regional" (local ?) : Pueblos southwest us ; concetration sud fr age de fer ; romanisation ; fragmentation territoires sud fr.
% RQ/Q : pertinence du concept de transition accross all these scales ? ~

% 10/12 modelisees (chap partie II)
% rq : deux pas model : Vacianni et Ducruet ...


%%%%%
%% Chap 3 :
% modeles spatio temporels
% (introduction) -> interet pedagogique ; methodologique (pde / abm) ; introductif ?

% - le fisher skelam (reaction diffusion, continu) [rq : paper Steele considerably extends basic model]

% - modele de Young : discret. ~ microsim model ?
%  lien math (rq : site du projet : pb perennisation ?)

% !!!! ndbp p100 : 3 repli : !! dangerous and wrong (give distroded impression to the reader ? despite menitonning chap 15)

% modele colodyn : comment on peut facielment etendre deux modeles simples, et faire du test d'hypotheses etc. [: ex strategies de deplacement peuvent etre comparees]
% rq : nombre de parameteres ! : deja enorme !




%%%%%%
%% Partie II


%%%
%% Chap 4 : sortie d'Afrique

% modele HUME
% -> modele agent : groupes humains (chasseurs ceuilleurs) / cellules.
% ~ model. incrementale (version simplifiee -> plus complexe)
% distribution spatiale de la biomasse.

% "explo systematique" : quelle plan experience ? nb params ? nb runs ? pourquoi 300 replis ?


%%%%
%% Chap 5 :
% Bantu / pygmees

% description contexte ; relations de causalite difficle a demeler en lien avec l'agriculture moteur.

% -> modele HUME2 extenion (adaptation) du HUME
%  - adaptation de la nature de l'environnem,ent (savane / foret)
%  - different parametres d'evolution pour les deux populations
%  - differents profils d'explotation des ressources (innnovation)
%  - effet d'anticipation (info sur etat cellules voisines)
%  + interaction entre les groupes

% quelques resultats : franchissement ou non 

% rq : disproportion entre la ocmplexite du modele (nbre params ?) et les explorations effectivement menees. Un parametre uniquement -> impression d'inacheve. [rq : general en abms social science ? devrait plus l'etre avec oml]


%%%%
%% Chap 6 : Pueblos

% important and fine archeological data in the study areas.

% Polity model for social dynamics (households ; groups : intra groups and inter groups dynamics) - game theory (public good games) ; 
% [mostly presenttaion fo the model]

% TODO Polity model : read in detail




%%%%
%% Chap 7 : Emergence des villes 

% modele SimpopLocal
% "transition irreversible" : concept interessant - sur quelle echelle de temps ? avec les nouveaux regimes (MCR puis ? ), seront-ce toujours des villes ?
% rq : endogene car multi-situe (pas une ville unique qui s'est diffusee) 
% def villes ? fonction cruciales.

% Independante donc necessaire ! => IDEE : necessite emergence des villes : lien complexite sociale (// papier base shesat ?)

% en reseau depuis leur emergence !

% description du simpoplocal.

% ! 500Mio simus ! OpenMOLE (le seul chapitre avec le 15 qui fait vraiment du systematique)
% rq : une des difficulte d'un tel interdisciplinarie : standardisation des reusltats / imposition niveau de robustesse, mais aussi reproducibilite ? : ouvre perspectives pour projets similaires !.



%%%%
%% Chap 8 : Mutations territoriales age fer sud Gaule

% regime : territoire eclate au continu (differentes maniere de s'alimenter et relation a l'environnement) (donnees archeologiques)

% Q : role de la colonisation dans evolution de la structure de l'habitat et des modes d'alim ?

% Transition : habitat petit et disperse => plus de sociabilite ; surplus, echanges economiques, elites

% modelisation : systemqie (diagrammes "sagittaux" - type uml)
% pas de modele de simulation : trop complique - complexe [!pas un argument] ; point de vues differents - complementaires.

% Annexes : 
 % - modele netlogo stylise par Crabtree : equilibre entre marchands et agents locaux (calibrer pour atteindre cet objectif) + preference des agents pour types de vins. 
 % - meta langauge sma base sur l'evolution des comportements [methode des cadrants : ?] . exprime relations entre agents (sans pouvoir les simuler ?)
 
 
 
 
%%%%
%% Chap 9 : Romanisation

% Structuration des territoires apres la conquete de la gaule. 

% role du fonctionnement social.
% / alimentation / transport et deplacements

% modelisation : diagrammes avant et apres la transition.

% theorie des jeux pour l'emergence de chef-lieux
% interactions entre rome et elites gauloises ; 

% trois strategies imperatores : force ; privilegier une cite ; installer colons - deux strategies elites gauloises : collaborer ; accepter installation de colons.
% => payoff matrix 8 x 3 x 2

% implementation : choix simultanes et sans communication ; un seul choix par joueur.

% graphe de transition entre etats. existence d'eq de Nash.
% strategie gain max : marche deterministe dans le graphe (?).


% rq : (discuté) : pas de dimension spatiale.


%%%%
%% Chap 10 : monde antique => monde medieval.

% transition "historique" 
% importance du recit (cas detude : languedoquecien)

% multiples echelles spatiales.

% role des eveches. (sorte de nouveau decoupage administratif)

% multiples facteurs de destructuration du cadre civique antique.

% structrues hierarchiques spatiales.

% modele conceptuel : entites spatiales = urbaines et rurales (multiples proprietes) ; entites sociales = pouvoirs locaux.
% dynamiques : pouvoir politique ; ecclesiastique. 
%  ! modele reste conceptuel - tres proche d'un abm toutefois.



%%%%
%% Chap 11 : polarisation et fixation habitat rural - 800-1100 (Touraine)

% - contexte : structuration de la societe feodale autour de l'an mil. d'un pouvoir centralise a decentralise ; habitat disperse a habitat polarise.

% description construction du modele : entites ; processus ; regles comportement
% -> sma. foyers paysans ; seignuers ; aggregat pop ; zones prelevement ; attracteurs (eglises et chateaux)

% initialization : espace stylise ; agents : donnes styilisees de la Touraine.
% seigneurs : creeent chateau ; 11 parametres. parametres caches sur les zones de prelevement ; grand/petits seigneurs - droits seigneuraux.
% paysans maximisent satisfaction ; min des trois composantes materielle, religieuse, protection.
% autres paramtres caches pour les attracteurs.
% [bien trop de paramteres ? ]

% simu : !!! 20 replis ! reproduit hierarchisation : objectif trop large.
% cl : calibration ?
% ---




%%%%%
%% Chap 12 - Vers des systemes de villes integres (France 18e 19e siecle)

% systemes de villes integres : auto-organises ; structure stable (en termes de specialisations economiques par exemple) ; co-evolution avec reseaux de communication ("retroaction positive centralite et accessibilite") -> hierarchisation des tailles de villes.

% -> etude de l'evolution des tailles des villes et reseau de communication : routes de poste. lien hierarchie urbaine et hierarchie routiere. 

% population : bases historiques ; numerisation des cartes des relais de poste. 

% avant transition : systemes regionaux, communication exceptionelle longue portee admin et grand commerce. pas de logique de reseau. 
% hierarchisation des travaux de renovation du reseau emergent.
% fin 18eme : relais de poste - maillage fin. de 600 relais en 1630 a 1500 en 1833.
% analogie reseau electrique : ??? (A. Franc ?)
% ponderation par diff altitude !
% simplification du graphe (algo classique)
% 
% cluster geo de resistance

% accroissement progressif de la vitesse.
% fort differentiel plaines / zones montagneuses.
% 
% debut 18e : fort decalage systeme de villes etabli depuis longtemps / vs reseau de communication recent.


% hierarchie du reseau : exposant de scaling pour les degres. 

% table d'association taille / classe d'accessibilite.

% emergence d'un reseau commercial.
% calcul betweenness (flux theoriques). 

% chgt des fonctions administrtatives des villes.

% modele de Reilly pour l'attractivite : pop / distance effective ^ beta
% -> carto "oursins" regionaux.
% transition : petits systemes regionaux en 1783 ; plus grosses polarisations en 1833
% (pb : que visuel ?)

% !!! stat pour coevol marche pas => donne pas resultats negatifs. (quels tests ; quels niveaux de significativite ; etc. )
% "echelles intermediaires ?" => villes moyennes . 



%%%%%%%
%% Chap 13 : regions metropolitaines polycentriques 


% augmentation des ehcanges materiels et immateriels entre les territoires.
% metropole : grande ville integree dans echange mondialises
% TODO (rq : redef selon une vue multi-scale - quest ce qui n'est pas integre auj ?)
% emergence des MCR : polycentrique et fortement integre echange economiques.

% transition (en cours) : de l'intermetropol a l'intra-MCR
% - role des mobilites
% portee spatiale a tps de transport donne : ++
% ex du bassin parisien. - integration des metropoles peripheriques secondaires. emergence de la grande couronne comme un espace central. pas centralise, cf Beauvais.

% TODO IDEE etudier cette transition avec un modele de systeme de ville ? (sans portee spatiale ontologisee peut etre complique ? )

% typologie des modes d'emergence des mcr

% implique de nouvelles formes de gouvernance territoriales. cooperatoin emergente. mutualisation des ressources. diversite des profils d'emergence en pratique. RhinRhur contreexemple ou gouvernance commune n'emerge pas. 

% scenarios pour l'etat final de la transition ?
%. : forte incertitude liee au prix futur de l'energie
%  3 scenarios stylisés : deconnexion ; integration intermediaire ; integration reussie dnas le cadre d'un progres technologique qui permet un maintien des pratique de mobilite (entre autre).

% description du modele lutecia. - application a config type. 
% experiment : !!! aucun sens stat !!!



%%%%%%%%%%
%% -- PARTIE III : synthese --
%%%%%%%%%%



%%%%%%%%
%% Chap 14 - Entre modeles et recits

% tjs des modeles, avec une dim temporelle. chaque transition liee a un recit historique. (recits stylisés).

% quest ce qui distingue alors le recit du modele ? 

% def modele ? "representation simplifiee" . role de l'oservateur (cf Minsky). 

% les modeles sont eux meme des recits

% comme pour le modele, version faibles et version fortes des recits. (construction et articulation des propositions)
% TODO Q : un modele generatif est il vraiment different des patterns generes ? => nouvelle approche du pom par classes d'equivalences. sur les arbres minimaux ontologiques, mettre une structure algebrique pas classses d'eq. comment quantifier la distance onto ? operation sur datasets ? distance de fit ? ...

% // analyse logiciste : decompose recit en modele generatif en quelque sorte ?

% lien avec domaines de connaissance. (lien entre donnees et theories). 

% modele permet de lier "le general et le particulier."
% rq : expression herissante, utilisee en mode cliche a tort et a travers.

% dynamique dans les modeles ; diachronicite ou synchronicite. espace et tps. 

% innovation et memoire.

 % emergence : faible et forte. 
 
 % outils : math et info (sma) 
 
 % differents rapports entre theories, modeles et donnees.
 
 % 
 
%%%%%%%
%% Chap 15 -- vers une recherche reproductible dans un cadre interdisc


% emergence d'une conscience autour reproducibilite

% rq : pas d'ouverture centralisee des codes source pour ce bouquin ? un peu un comble ?

% reproductibilite en SHS
% mesures ou raisonnement sci. ?
% equifinalite : cumulutativite des connaissances (Pumain)
% 

% Transfert du cadre conceptuel
% rq : arbitraire et un peu creux finalement ? (seuil arbitraire du nombre de dimensions ? )
% (semble montrer def bof en fait ? )

% Moyens technique pour reproductibilite des modeles
% etapes successive vers repetabilite.
% openmole pour replicabilite exploration modele.
% TODO idee en lien avec plugin git gui : genre metafig avec hash commit + fichiers de metadata ? 
% ex d'application modele de Young. 
% cl : technique et conceptuel (rq : cf cadre de connaissance !)


%%%%%%
%% Chap 16 -- Points de vue ontologiques sur les transitions

% ~> explicitation des ontologies sous-jacentes. 

% cadre pour passer de l'empirique au formel via interaction des domaines conceptuels.
% ontologie propres a chaque discipline

% ontoolgie : cadre de description explicite visant a eliminer ambiguites.

% -> entites pertinentes et operations qui peuvent etre operees dessus.

% exemples de choix ontologiques explicites ou implicites.

% co-construction des modeles : illustration sur les differentes transitions.

% differents point de vues ontologiques : uml info ; thematqiue explicitation des objets ; philosophe ontologie.

% synthese de chaque transition.

% diversite des approches de omdelisation voulue 
% rq : un peu de l'applied perspectivism ! mais sur des cas differents. auraient du doubler les groupes de travail pour chaque transition.


%%%%%%
%% Chap 17 -- retours interdisc sur le projet

% retour d'experience sur dialogiue interdisc et coconstruction.

% identifier transition discussion en elle meme.

% echange entre les differents groupes sur les differentes transitions.

% importance des data dans la relation thematique / model.

% le modelisateur comme mediateur ?

% + annexes : points de vue particuliers.



















	
	
	
	
	
	
	
	
	
	
	
	
\end{document}